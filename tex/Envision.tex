In order to understand the concept of horizontal security, we must first examine
the current state of security and its goals. For the purposes of this work, we
will examine security through the traditional scheme of Confidentiality,
Integrity and Availability, as is common. According to NIST,

\begin{quotation}
\textit{A loss of confidentiality is the unauthorized disclosure of 
information... A loss of integrity is the unauthorized modification or
destruction of information... A loss of availability is the disruption of access
to or use of information or an information system.}~\cite{pub2004standards}
\end{quotation}

Thus, from these definitions we get that confidentiality is the protection of 
information from unauthorized disclosure, integrity is the protection of
information from unauthorized modification or destruction, and availability is
the protection from the disruption of access to or use of information or an
information system.

Present in two of these three definitions is the concept of authorization. If an
authorized party reads, modifies, or destroys information, the definitions imply
that there is no breach. Thus, whether or not the security of an organization has
been breached depends highly on whether or not an individual is authorized to
perform an action. This decision as to whether or not a person is authorized is
heavily informed by an organization's security policy. However, this leads us to
follow up questions: who writes these security policies, and how are the
decisions of what they should contain made? How do we design authorization when
authority is distributed across the entire collective?

\subsection{Defining Horizontal Security}
\label{sec:definition}

In the current approaches to Cybersecurity, these decisions are often made or
guided by the desires of those on top of the hierarchy of an organization. If
the CISO of an organization does not agree with the access control decisions of
an organization, they will insist that the managers below them see that it is
changed. This will trickle down the hierarchy until it is implemented. Though
individuals at the lower levels of a hierarchy may have some decision making
power in how it is implemented, they typically do not have decision making power
in the questions mentioned above. Further questions arise when we consider the
structures of organizations with flexible memberships, who operate on principles
of open participation (plein-air citizens assemblies, etc.).

This concept typically holds true not just with access control and
authorization, but with all aspects of an organization's security policy.
Decisions about how infringements are dealt with or the scope of the policies
are also made in this manner. Decisions about how these policies are enforced,
and what security mechanisms are used, are also made up top, with individuals at
the bottom left to enact the desires passed on to them.

We see, then, that hierarchy introduces itself in multiple dimensions of
security. These dimensions include the different decision-making processes
of security. This implies that the concepts of confidentiality, integrity, and
availability, which are informed by an organization's security policy, may not
need to change under a concept of horizontal security. Instead, it is possible
that horizontal security may be achieved by shifting policy making powers,
implementation decisions, and enforcement powers, and other dimensions of
security away from individuals and towards democratic, participatory processes.

%\begin{definition}[Horizontality of Policy]
%Consider a set of stakeholders $ S \neq \emptyset$. There exists a set of
%security policymakers, $ p $ such that $ p \subseteq S $ and $p \neq 
%\emptyset$. We say that the horizontality of policy making $h_{policy}$ of the
%organization is defined as $ h_{policy} = \frac{|p|}{|S|} $.\\
%\noindent\rule{.48\textwidth}{0.4pt}
%\end{definition}
%
%\begin{definition}[Horizontality of Implementation]
%Consider a set of stakeholders $ S \neq \emptyset$. There exists a set of
%security implementers, $ i $ such that $ i \subseteq S $ and $i \neq 
%\emptyset$. We say that the horizontality of implementation $h_{implementation}$
%of the organization is defined as $ h_{implementation} = \frac{|i|}{|S|} $.\\
%\noindent\rule{.48\textwidth}{0.4pt}
%\end{definition}
%
%\begin{definition}[Horizontality of Enforcement]
%Consider a set of stakeholders $ S \neq \emptyset$. There exists a set of
%security enforcers, $ e $ such that $ e \subseteq S $ and $e \neq \emptyset$.
%We say that the horizontality of enforcement $h_{enforcement}$ of the
%organization is defined as $ h_{enforcement} = \frac{|e|}{|S|} $.\\
%\noindent\rule{.48\textwidth}{0.4pt}
%\end{definition}
%
%\begin{definition}[Horizontality of Security]
%We say the horizontality of an organization's security $h$ is defined as $h = 
%h_{policy} \times h_{implementation} \times h_{enforcement}$.\\
%\noindent\rule{.48\textwidth}{0.4pt}
%\end{definition}

One important aspect of this concept is the fact that the horizontality of 
different dimensions of security -- policy making, implementation, enforcement,
and more -- have a compounding effect. If one area is highly hierarchical and
centralized, it has a profound effect on the hierarchy and centralization of the
organization as a whole. For example: if the implementation of a security policy
(through choosing security mechanisms, implementing firewall rules, etc.) is
heavily centralized and hierarchical, then the member(s) of the organization who
implement the policies can decide how to implement security mechanisms to ensure
policies are followed, or worse still, whether to implement any security
mechanisms at all. This centralization leads to an imbalance of power similar to
the one discussed by Kavada regarding activism and social
media~\cite{kavada2020counterpublics}.

Another implication of the fact that the horizontality of security is
multi-dimensional is that horizontality is not a binary trait, but rather a
spectrum between entirely dictatorial and fully horizontal. In practice, most
organizations likely do not sit at either extreme of this spectrum, or even
occupy a fixed space on this spectrum. Rather, organizations are likely to have
dynamic levels of horizontality. In traditional, hierarchical organizations,
security is likely more skewed towards a lower amount of horizontality, and is
likely to not be too dynamic. Decisions are made by managers, who may or may not
consider the input of their employees when making decisions. In worker
cooperatives,  activist groups, and other traditionally less hierarchical
organizations, security is more likely skewed towards a higher amount of
horizontality, and may be much more dynamic in nature. An interesting first work
towards understanding horizontal security may exist in classifying, comparing,
and contrasting the current horizontality of security in different types of
organizations. This idea, and other potential future works, are further examined
in Section~\ref{sec:futurework}.

% TODO: Include here the lack of ability to be horizontal when hierarchical
% tools are used, and vice-versa, the lack of ability to be hierarchical when
% horizontal tools are used. Can we develop technology that allows us to do
% both when they are needed?
\subsection{Horizontality, Tools, and Techniques}
In Section~\ref{sec:definition} we discussed horizontality of security as being
multidimensional and dynamic, and relating to the different decision making
processes that an organization has around securit. Thus, we see that
horizontality is related to security policy decisions, as the participation in
these decision making processes informs what power individuals in the
organization have to say how they wish to secure their operations.

However, this does not mean that the influence of hierarchy does not affect the
tools and techniques used in Cybersecurity, and that some tools and techniques
do not imply or impose hierarchy. For example, most operating systems
tend to be built around the idea of the singular administrator or a small set of
administrators setting up the system. This idea makes sense from a  hierarchical
organization's perspective. After all, if the individuals higher in the
hierarchy have more privileges, that simply reflects their decision-making 
power within the organization. From a horizontal security perspective, however,
this method of system administration is detrimental. If a singular administrator
decides that they do not like the policy decisions of the organization, they may
be able to use their administrative power to stop the organization from taking
the democratically selected action. When this occurs, there may not be another
individual with more privileges to override this decision. Thus, the very
concept of an administrator with the power to shut out other administrators is a
threat to a horizontally controlled organization.

On the other hand, there do exist systems that were designed with a more
horizontal structure in mind. One example of this is the Bitcoin 
cryptocurrency~\cite{nakamoto2019bitcoin}. This cryptocurrency was developed to
allow for financial transactions between two individuals without the need for a
trusted third party to detect and prevent double spending. By design, the
network prevents the modification of past blocks of the Blockchain unless there
exists collusion of 51\% of the miner network. This decision was made
specifically to make it difficult for malicious actors in the network to perform
malicious activities, such as stealing bitcoin or deleting past transactions.
The implication of this design is that Bitcoin cannot be centrally 
regulated~\cite{tu2015rethinking}, at least in theory . If we re-frame the
concept of collusion as cooperation via democratic vote, we see that this design
allows for regulation of the Bitcoin network in a horizontal way. Thus, if a
transaction was deemed to be against a security policy by the majority of the
network, it is possible to remove it.

Though this is more horizontal policy enforcement than other tools promote, it
still has many issues. The first issue is that of defining the stakeholders of
Bitcoin: are only miners stakeholders, or are users stakeholders as well? If we
define the stakeholders are only the miners of Bitcoin, we see that this
approach reaches perfect horizontality. All miners participate in the decision
to perform a modification on the Blockchain, whether the vote yes or no. If we
consider users to also be stakeholders, however, we see that this system is no
longer as horizontal. Users do not get a vote or say in whether or not a
transaction gets removed or modified. 

Another issue with this approach is its competitive nature. Rather than
cooperating miners attempting to create, implement, and enforce transaction
policies, we have competing miners interested in their own profit. One
implication of this competitive focus is that there is no mechanism to vote in
Bitcoin; rather, what is referred to as ``regulation'' in this work is seen by
the Bitcoin community as an attack. However, the design property of requiring
51\% of participating members to agree in order to make a change on the system
may still be useful to inspire future horizontal system designs. Worse still,
this competative nature tends to centralize the network over time. Bitcoin's
Proof of Work algorithm tends to favor those who can afford highly specialized
software, eventually leading to a centralization of the network into the hands
of those who are wealthy enough to control large amounts of machines
specifically made for mining Bitcoins. Proof of Stake, another consensus
algorithm for Blockchain, bypasses this middle step and directly benefits those
who control larger amounts of monetary stake in the network.

% TODO: Add something about the DOA hack.

Despite these issues, the security design of Blockchain is still more horizontal
than typical designs. However, Blockchain is no the only security technology
with a less hierarchical vision and structure. Perhaps the most well known
example of a security system meant to be horizontal by design is the PGP Web of
Trust design. This design was created to solve the key distribution problem
without the need for a centralized, trusted key distribution authority. It
allows individuals to make trust decisions about the ownership of a key based on
validating the key ownership in person or over a trusted secure channel. Once a
key is validated, the keys that are signed by that individual's keys are also
trusted, though to a lesser extent. This design was meant to be a horizontal
representation of trust: one trusts the people they verify in person, and then
trusts the people attested to by their trusted peers, but to a lesser degree.
However, this scheme suffered from a large number of usability and user
experience problems. {\color{red} TODO: Cite me.} However, if a method like web
of trust or derived from web of trust could be improved to be more usable, it
could serve as a building block for applications and systems that take a
horizontal approach to security.

%TODO: Flesh the following paragraph out. Talk about the cipherpunks.
In addition to the Blockchain approach used by Bitcoin, and to the web of trust
used by PGP, many other potential building blocks of horizontal security already
exist. These include other consensus algorithms such as Practical Byzantine
Fault Tolerance (PBFT)~\cite{castro1999practical}, Paxos~\cite{lamport2019part}, 
Raft~\cite{ongaro2013search}, and more. In addition to consensus algorithms,
secret sharing schemes like Shamir Secret Sharing~\cite{shamir1979share}, 
Blakley's Secret Sharing~\cite{blakley1993linear}, and Verifiable Secret 
Sharing~\cite{chor1985verifiable} schemes can be used for distributed secret
management in organizations. Applications like OAuth~\cite{leiba2012oauth} can
be used to grant minimal and ephemeral access to organization accounts to
members of the organization who need it, without revealing a password. 
Capability based schemes and systems can be used to allow for transfer of
control over resources based on the will of the collective. Additionally, other
yet undeveloped primitives, techniques, and tools may be needed as well. Future
work can address these needs, as described in Section~\ref{sec:futurework}.

\subsection{COLBAC: Collective Based Access Control}
\label{sec:colbac}
As discussed in previous sections, the choice of certain software, protocols, or
techniques such as X509 certificates or PGP web of trust has implications in the
level of horizontality possible for systems built on top of those software,
protocols, or techniques. How, then, can we build a foundation such that
organizations of different horizontality can use the same foundation and arrive
at much differently structured organizations?

In this section we describe our Collective Based Access Control, or COLBAC: our
approach to a dynamic horizontality access control system. We begin by
describing its requirements, focusing on the dynamic horizontality that
separates COLBAC from other access control systems. We then define the system
itself. Finally, we discuss the properties of this system and the limitations of
this system.

\subsubsection{System Requirements}
\mbox{}\\
COLBAC is aimed at addressing a novel requirement in cybersecurity research:
access control and authorization given an organization with dynamic levels of
horizontal control. Though previous approaches exist for hierarchical access
control (such as MAC, RBAC, etc.), to our knowledge no access control model
exists for organizations of dynamic and flexibile horizontality. To realize
this, our solution must be able to be flexible in terms of horizontality. Said
another way, our system must not assume or define a pre-determined threshold for
horizontal control, i.e., it must not assume majority, or super-majority, or 
full consensus is the preferred method of democratic participation. Instead,
the threshold must be configured by the collective itself, and must be able to
be changed when necessary. This allows for rapid temporary centralization of the
system to respond to crises, or to perform a task that requires expertise that
few members of the collective contain. However, these moments of centralization
must be quick to expire and easy to override in order to prevent abuse of
centralized power. Said another way, it must always be easy for the collective
to return to more horizontal control.
\begin{enumerate}
\item full transparency of actions taken on behalf of the collective.
\end{enumerate}

\subsubsection{System Design}
\mbox{}\\
COLBAC presents a solution to access control that relies around the collective.
However, as will be discussed later in this section, not all objects on the
system will need to be collectively controlled or administered. As such, we
define three distinct \textit{spheres} of the system, or areas that require
different approaches to access control. These spheres are the \textbf{Collective
Sphere}, the \textbf{User Spheres}, and the \textbf{Immutable Sphere}.

In order to achieve different degrees of horizontality, there must be a portion
of the system that is controlled not by any individual user of the system, but
by some democratic process of the users of the system. We call this portion of
the system the \textbf{Collective Sphere}, as it contains programs, files, and
other resources only accessed based on collective authorization. In any
horizontal system, the administrative functions of the operating system would
need to exist within this portion of the system to allow for true collective
control.

However, not everything should be directly managed by the collective. Individual
users may have their own files and programs they intend to use only in a way
that it does not effect other users of the system or the resources of the
collective\footnote{We can think of these as the home directories of users in
modern Unix-like systems}. These speheres, called the \textbf{User Speheres},
can use traditional DAC systems like modern Unix-like systems without effecting
the horizontality of the system as a whole.

Finally, to have meaningful control of the system we must have transparency. To
achieve this, a system must have an \textbf{Immutable Sphere}, or a portion of
the file system and programs that cannot be altered once written to, including
by democratic control. This allows for the system to provide append only logs
that are vital to maintaining collective control, as described later in this
section.

\begin{enumerate}
%\item Different Spheres
%    \begin{itemize}
%    \item user sphere
%    \item collective sphere (sudo)
%    \item immutable sphere (cannot be altered, can be read)
%    \end{itemize}
\item Registration
    \begin{itemize}
    \item at least 3 users
    \item each user has an offline key.
    \item specify beginning fraction required for collective action
    \end{itemize}
\item Token: Draft Phase
    \begin{itemize}
    \item User drafts a proposal. Two types, action token and delegation token.
    \item See subsection for proposal design.
    \item user signs the proposal with their key.
    \end{itemize}
\item Token: Petition Phase
    \begin{itemize}
    \item Pass along token to reference monitor, reference monitor passes to
    other users.
    \item If a user agrees, signs and returns to reference monitor.
    \item If a user disagrees, signs a disagreement and returns to reference
    monitor.
    \item User can choose not to vote. Signs a blank vote and returns to
    reference monitor.
    \end{itemize}
\item Token: Action Phase
    \begin{itemize}
    \item When petition phase is over, the number of signatures is compared to
    the fraction.
    \item If >=, perform action, log.
    \item Else, just log.
    \end{itemize}
\end{enumerate}

\subsubsection{Types of Tokens}
\mbox{}\\
\begin{enumerate}
\item Action Token
\item Delegation Token (Working Group Token)
\item Emergency Token
\end{enumerate}

\subsubsection{Properties of COLBAC}
\mbox{}\\
\begin{enumerate}
\item flexible and dynamic horizontality through configurable fraction.
\item full transparency of actions taken.
\end{enumerate}

\subsubsection{Limitations of COLBAC}
\mbox{}\\
\begin{enumerate}
\item USABILITY!!!!!!!
\end{enumerate}


% Access Control:
% La-Padula, Biba, strictly hierarchical
% DAC based on individualistic regulation of files, overwritable by admin on
%    most systems.
% MAC or RBAC seem to be desirable for horizontal security, but how do we update
%    policies?

% Bitcoin:
% attempt at horizontal, democratic control of currency.
% Regulatable with 51%.
% suffers from centralization as it goes on due to its competative nature.

%\begin{enumerate}
%\item Case study: Code review.
%\item Case study: Peer review.
%\end{enumerate}

%Conclusion: Security is purely horizontal when the policies and enforcement are
%handled via direct democratic vote. Horizontality is a spectrum, ranging from
%hierarchical (one person with decision making power) to purely horizontal.
%Examples of spots on the spectrum include:
%\begin{itemize}
%\item direct democracy / consensus cooperatives
%\item Random review (i.e. code review)
%\item random/elected representatives w/ early revocation
%\item random/elected representatives w/o early revocation
%\item appointed representatives
%\item hierarchical
%\end{itemize}

%Very few spots on this spectrum are represented in current techniques and
%techologies. Mostly the hierarchical approach is developed because of its
%continued use in business and government. As that changes, cybesecurity needs to
%be ahead of the curve and create technologies for the organizations and
%businesses that sit on the other portions of this spectrum.

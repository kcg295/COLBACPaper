In order to understand the concept of horizontal security, we must first examine
the current state of security and its goals. For the purposes of this work, we
will examine security through the traditional scheme of Confidentiality,
Integrity and Availability, as is common. According to NIST,

\begin{quotation}
\textit{A loss of confidentiality is the unauthorized disclosure of 
information... A loss of integrity is the unauthorized modification or
destruction of information... A loss of availability is the disruption of access
to or use of information or an information system.}~\cite{pub2004standards}
\end{quotation}

Thus, from these definitions we get that confidentiality is the protection of 
information from unauthorized disclosure, integrity is the protection of
information from unauthorized modification or destruction, and availability is
the protection from the disruption of access to or use of information or an
information system.

Present in two of these three definitions is the concept of authorization. If an
authorized party reads, modifies, or destroys information, the definitions imply
that there is no breach. Thus, whether or not the security of an organization has
been breached depends highly on its security policy. However, this leads us to
follow up questions: who writes these security policies, and how are the
decisions of what they should contain made? How do we design authorization when
authority is distributed across the entire collective?

\subsection{Defining Horizontal Security}
\label{sec:definition}

In the current approaches to Cybersecurity, these decisions are often made or
guided by the desires of those on top of the hierarchy of an organization. If
the CISO of an organization does not agree with the access control decisions of
an organization, they will insist that the managers below them see that it is
changed. This will trickle down the hierarchy until it is implemented. Though
individuals at the lower levels of a hierarchy may have some decision making
power in how it is implemented, they typically do not have decision making power
in the questions mentioned above. Further questions arise when we consider the
structures of organizations with flexible memberships, who operate on principles
of open participation (plein-air citizens assemblies, etc.).

This concept typically holds true not just with access control and
authorization, but with all aspects of an organization's security policy.
Decisions about how infringements are dealt with or the scope of the policies
are also made in this manner. Decisions about how these policies are enforced,
and what security mechanisms are used, are also made up top, with individuals at
the bottom left to enact the desires passed on to them.

We see, then, that hierarchy introduces itself in the decision-making processes
of security. That is, the concepts of confidentiality, integrity, and availability
may not need to change under a concept of horizontal security. Instead, it is
possible that horizontal security may be achieved by shifting policy making
powers, implementation decisions, and enforcement powers away from individuals
and towards democratic, participatory processes. A more concrete definition of
horizontal security follows.

\begin{definition}[Horizontality of Policy]
Consider a set of stakeholders $ S \neq \emptyset$. There exists a set of
security policymakers, $ p $ such that $ p \subseteq S $ and $p \neq 
\emptyset$. We say that the horizontality of policy making $h_{policy}$ of the
organization is defined as $ h_{policy} = \frac{|p|}{|S|} $.\\
\noindent\rule{.48\textwidth}{0.4pt}
\end{definition}

\begin{definition}[Horizontality of Implementation]
Consider a set of stakeholders $ S \neq \emptyset$. There exists a set of
security implementers, $ i $ such that $ i \subseteq S $ and $i \neq 
\emptyset$. We say that the horizontality of implementation $h_{implementation}$
of the organization is defined as $ h_{implementation} = \frac{|i|}{|S|} $.\\
\noindent\rule{.48\textwidth}{0.4pt}
\end{definition}

\begin{definition}[Horizontality of Enforcement]
Consider a set of stakeholders $ S \neq \emptyset$. There exists a set of
security enforcers, $ e $ such that $ e \subseteq S $ and $e \neq \emptyset$.
We say that the horizontality of enforcement $h_{enforcement}$ of the
organization is defined as $ h_{enforcement} = \frac{|e|}{|S|} $.\\
\noindent\rule{.48\textwidth}{0.4pt}
\end{definition}

\begin{definition}[Horizontality of Security]
We say the horizontality of an organization's security $h$ is defined as $h = 
h_{policy} \times h_{implementation} \times h_{enforcement}$.\\
\noindent\rule{.48\textwidth}{0.4pt}
\end{definition}

One important aspect of this definition is the fact that the horizontality of 
different factors of security -- policy making, implementation, and enforcement
-- are multiplied together rather than averaged. This decision was made to
reflect the compounding control that centralization gives: if the implementation
of a security policy is heavily centralized and hierarchically, then the
member(s) of the organization who implement the passed policies can decide
how to implement security mechanisms to ensure policies are followed, or worse
still, whether to implement any security mechanisms at all. This centralization
leads to an imbalance of power similar to the one discussed by Kavada regarding
activism and social media~\cite{kavada2020counterpublics}.


This definition, even with the compounding effect of centralization in any of
the three areas, implies that horizontality is not a binary trait, but rather a
spectrum between entirely dictatorial (defined as $h = \frac{1}{|S|}$) and fully
horizontal (defined as $h = 1$). In practice, most organizations likely do not
sit at either extreme of this spectrum. In traditional, hierarchical
organizations, security is likely more skewed towards a lower fraction of
horizontality, but not equal to $\frac{1}{|S|}$. In worker cooperatives, 
activist groups, and other traditionally less hierarchical organizations,
horiztonality is likely more skewed towards, but not arriving at, $1$. An
interesting first work towards understanding horizontal security may exist in
measuring, comparing, and contrasting the current horizontality of security in
different types of organizations. This idea, and other potential future works,
are further examined in Section~\ref{sec:futurework}.

\subsection{Horizontality, Tools, and Techniques}
In Section~\ref{sec:definition} we defined horizontality of security as the
fraction of policymakers over stakeholders of an organization. Thus, we see that
horizontality is related to security policies more than the tools used to 
enforce them. 

However, this does not mean that the influence of hierarchy does not affect the
tools and techniques used in Cybersecurity. For example, most operating systems
tend to be built around the idea of the singular administrator or a small set of
administrators setting up the system. This idea makes sense from a  hierarchical
organization's perspective. After all, if the individuals higher in the
hierarchy have more privileges, that simply reflects their decision-making 
power within the organization. 

From a horizontal security perspective, however, this method of system 
administration may be detrimental. If a singular administrator decides that
they do not like the policy decisions of the organization, they may be able to
use their administrative power to stop the organization from taking the
democratically selected action. When this occurs, there may not be another
individual with more privileges to override this decision. Thus, the very
concept of an administrator with the power to shut out other administrators is a
threat to a horizontally controlled organization. 

On the other hand, there do exist systems that were designed with a more
horizontal structure in mind. One example of this is the Bitcoin 
cryptocurrency~\cite{nakamoto2019bitcoin}. This cryptocurrency was developed to
allow for financial transactions between two individuals without the need for a
trusted third party to detect and prevent double spending. By design, the
network prevents the modification of past blocks of the Blockchain unless there
exists collusion of 51\% of the miner network. This decision was made
specifically to make it difficult for malicious actors in the network to perform
malicious activities, such as stealing bitcoin or deleting past transactions.
The implication of this design is that Bitcoin cannot be centrally 
regulated~\cite{tu2015rethinking}. If we re-frame the concept of collusion as 
cooperation via democratic vote, we see that this design allows for regulation
of the Bitcoin network in a horizontal way. Thus, if a transaction was deemed
to be against a security policy by the majority of the network, it is possible
to remove it.

Though this is more horizontal policy enforcement than other tools promote, it
still has many issues. The first issue is that of defining the stakeholders of
Bitcoin: are only miners stakeholders, or are users stakeholders as well? If we
define the stakeholders are only the miners of Bitcoin, we see that this
approach reaches perfect horizontality. All miners participate in the decision
to perform a modification on the Blockchain, whether the vote yes or no. If we
consider users to also be stakeholders, however, we see that this system is no
longer as horizontal. Users do not get a vote or say in whether or not a
transaction gets removed or modified. 

Another issue with this approach is its competitive nature. Rather than
cooperating miners attempting to create, implement, and enforce transaction
policies, we have competing miners interested in their own profit. One
implication of this competitive focus is that there is no mechanism to vote in
Bitcoin; rather, what is referred to as ``regulation'' in this work is seen by
the Bitcoin community as an attack. However, the design property of requiring
51\% of participating members to agree in order to make a change on the system
may still be useful to inspire future horizontal system designs. Worse still,
this competative nature tends to centralize the network over time. Bitcoin's
Proof of Work algorithm tends to favor those who can afford highly specialized
software, eventually leading to a centralization of the network into the hands
of those who are wealthy enough to control large amounts of machines
specifically made for mining Bitcoins. Proof of Stake, another consensus
algorithm for Blockchain, bypasses this middle step and directly benefits those
who control larger amounts of monetary stake in the network.

Despite these issues, the security design of Blockchain is still more horizontal
than typical designs. However, Blockchain is no the only security technology
with a less hierarchical vision and structure. Perhaps the most well known
example of a security system meant to be horizontal by design is the PGP Web of
Trust design. This design was created to solve the key distribution problem
without the need for a centralized, trusted key distribution authority. It
allows individuals to make trust decisions about the ownership of a key based on
validating the key ownership in person or over a trusted secure channel. Once a
key is validated, the keys that are signed by that individual's keys are also
trusted, though to a lesser extent. This design was meant to be a horizontal
representation of trust: one trusts the people they verify in person, and then
trusts the people attested to by their trusted peers, but to a lesser degree.
However, this scheme suffered from a large number of usability and user
experience problems. {\color{red} TODO: Cite me.} However, if a method like web
of trust or derived from web of trust could be improved to be more usable, it
could serve as a building block for applications and systems that take a
horizontal approach to security.

In addition to the Blockchain approach used by Bitcoin, and to the web of trust
used by PGP, many other potential building blocks of horizontal security already
exist. These include other consensus algorithms such as Practical Byzantine
Fault Tolerance (PBFT)~\cite{castro1999practical}, Paxos~\cite{lamport2019part}, 
Raft~\cite{ongaro2013search}, and more. In addition to consensus algorithms,
secret sharing schemes like Shamir Secret Sharing~\cite{shamir1979share}, 
Blakley's Secret Sharing~\cite{blakley1993linear}, and Verifiable Secret 
Sharing~\cite{chor1985verifiable} schemes can be used for distributed secret
management in organizations. Applications like OAuth~\cite{leiba2012oauth} can
be used to grant minimal and ephemeral access to organization accounts to
members of the organization who need it, without revealing a password. 
Capability based schemes and systems can be used to allow for transfer of
control over resources based on the will of the collective. Additionally, other
yet undeveloped primitives, techniques, and tools may be needed as well. Future
work can address these needs, as described in Section~\ref{sec:futurework}.
% Access Control:
% La-Padula, Biba, strictly hierarchical
% DAC based on individualistic regulation of files, overwritable by admin on
%    most systems.
% MAC or RBAC seem to be desirable for horizontal security, but how do we update
%    policies?

% Bitcoin:
% attempt at horizontal, democratic control of currency.
% Regulatable with 51%.
% suffers from centralization as it goes on due to its competative nature.

%\begin{enumerate}
%\item Case study: Code review.
%\item Case study: Peer review.
%\end{enumerate}

%Conclusion: Security is purely horizontal when the policies and enforcement are
%handled via direct democratic vote. Horizontality is a spectrum, ranging from
%hierarchical (one person with decision making power) to purely horizontal.
%Examples of spots on the spectrum include:
%\begin{itemize}
%\item direct democracy / consensus cooperatives
%\item Random review (i.e. code review)
%\item random/elected representatives w/ early revocation
%\item random/elected representatives w/o early revocation
%\item appointed representatives
%\item hierarchical
%\end{itemize}

%Very few spots on this spectrum are represented in current techniques and
%techologies. Mostly the hierarchical approach is developed because of its
%continued use in business and government. As that changes, cybesecurity needs to
%be ahead of the curve and create technologies for the organizations and
%businesses that sit on the other portions of this spectrum.

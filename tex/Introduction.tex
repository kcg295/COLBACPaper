The struggle of power between centralized, hierarchical structures and
decentralized, horizontal structures has existed for millennia. This struggle
has expressed itself in all forms of life: politics, economics, social
interaction, and more. Recent events all around the globe have demonstrated the
continuance of this struggle: while wealth inequality increases, surveillance
technologies are deployed more widely, and political power centralizes in the
hands of political families, other elements of society demonstrate a desire for
more inclusive, participatory, and fair social structures. Examples of this can
be seen in the Arab Spring movements, which demanded more democratic
participation in governments of the Arab world, and the Occupy Movement, which
demanded more democratic participation in economic and political policies. More
recently, in Hong Kong recent protests have demanded the right to participate in
their own government through democratic vote. In the Movement for Black Lives,
some Black Lives Matter chapters are demanding more than just increased police
oversight and funding reallocation, but community control of police, including
access to relevant local, state, and federal law enforcement agency
information~\cite{m4bl2021}. In the economic sector, worker cooperatives and
worker participation in their place of employment is becoming more prevalent,
and research is beginning to demonstrate economic advantages to cooperative
organization of the
workplace~\cite{jackall19846,wright2014worker, lindenfeld1982workplace}.
% XXX: if you want to go shitlib, we can argue for the presence of the PRO act in the US
Even some schools are beginning to include the democratic participation of the
school children to their decision making processes, encouraging engagement and
investment in the child's own education~\cite{pacheco2008escola}. In many
sectors of life, the centralization and strict hierarchy of current organization
is being challenged by a demand for more participatory structures.

%This shift in organization has potential implications in how organizations
%interact internally, with each other, and with society as a whole. Whereas the
%interactions between hierarchical organizations are overwhelmingly guided by
%the visions of the individuals who are on top of their respective organization's
%hierarchies, interactions of more horizontal organizations are more complex, and
%are meant to reflect the will of the entire group that make up the organization.
%In addition, responsibilities in more horizontal organizations tend to shift.
%For example, in worker cooperatives individuals often shift between different
%job positions in order to learn how the company as a whole functions, and to
%keep their work fresh and 
%interesting~\cite{jackall19846, wright2014worker, lindenfeld1982workplace}.
%To complicate matters further, different organizations have different, and
%sometimes dynamic, levels of  horizontality~\cite{wright2014worker}. Some decide
%to shift all decision making power to the collective, making approval for all
%important actions require democratic votes or consensus. Others select
%representatives or working groups to delegate specific tasks, allowing for
%quicker actions and responses to events, but within a power structure that
%allows for the revocation of those who misuse their power. Some organizations
%take a hybrid approach, requiring democratic participation of the whole
%organization for some issues, and electing working groups for others. Some
%organizations sway between all of the above listed approaches.

Despite the rapid demand of this changed structure of organization, 
Cybersecurity tools, technologies and techniques continue to remain highly 
hierarchical. Security policies are written by a security team or imported from
pre-existing (and sometimes external) templates and imposed on the rest of the
organization, with at most the limited oversight and approval of a manager or
higher-up. Even still, not every member of a security team participates
in deciding what these policies are meant to achieve. This decision is often
made by a manager, who ensures that the policies do not get in the way of the
visions of the individuals on the top of their hierarchies. In developing
security policies and implementing security technologies, the resulting
technologies and policies reflect political decisions made during their
creation~\cite{winner1980artifacts}. The resulting policies and technologies
then themselves have political implications that affect the individuals,
organizations, and societies that use them. Since most security tools have their
roots in hierarchical entities such as military or corporate business, these
tools tend to reflect the assumptions of hierarchy that exist in those
communities. While this may make sense for the military or corporate sectors,
these technologies are not the best fit for horizontal and participatory sectors
such as the worker cooperative sector, the activism sector, and the nonprofit
sector.

One example of this is the access to organization controlled secrets. Secrets
such as organizational passwords to third-party services are often trusted to
one individual or a small group in an organization, but easily recovered by an
administrator or other individual higher in the hierarchy in the case that the
individual responsible leaves the organization or performs a violation. However,
when an organization becomes less hierarchical, the decision of who should have
access to such secrets becomes more difficult to make. If everyone should have
access to the secrets, such as passwords, then any insider can change the
passwords and cause denial of service for the organization. If only a few people
have access to the password, then a hierarchy forms based on one's access to
confidential organizational material. This can lead to events like password 
wars~\cite{kavada2020counterpublics}, or issues like the digital vanguard
problem~\cite{gerbaudo2017social}.

A case in point: one of the organizations we have observed in our ethnographic
research was a democratic trade union, which had recently transitioned from a
more hierarchical and bureaucratic organizational structure to one which
involved more transparent decision-making and meetings, with direct
participation by members (meetings were open to all, membership became less
exclusive, and contract bargaining was open to all members, among other
reforms.) This change came about through years of grass-roots organizing within
the union, and was implemented through a democratic election in which a new
slate of representatives took power within the union. In this model -- neither
purely direct nor purely representative democracy -- a committee affiliated with
the ruling caucus was charged with handling communications to the members,
through consensus-based decision-making. For this purpose, access to the union's
email contacts and digital archives had to be transferred to the committee. 

Here is where the trouble began. Although a democratic process had put the
committee in power, and although the committee shared decision-making power
equally across all members, access to digital communications and member lists
was controlled by single passwords. In the past, these communications had been
handled by one or a few people who had the passwords and who worked within a
hierarchical structure, implementing the wishes of their superiors. These actors
refused to share the passwords with the new committee, and a great struggle
ensued, for weeks. During this period, communications to the membership were
sent illegitimately and by those who previously held power. A great deal of time
and energy was spent coming to a compromise, and forcing the former caucus to
share access. Here we see a clear case wherein the digital security design does
not respond to or mirror changes in the political structure of a democratic
collective. This can be the site of injustice, struggle, and a place where a lot
of time, energy, and credibility is lost. A more distributed version of access
control could have averted this situation.

Similar stories arise in other research at the intersection of communications
and social movements: Anastasia Kavada, in her research on the Occupy movement,
describes multiple cases, in different cities, of administrators being locked
out of facebook pages or hijacking the group’s twitter account to tell their own
narrative~\cite{kavada2015creating} Similarly, Paulo Gerbaudo theorizes what he
calls a ``digital vanguard'' in non-hierarchical social movements, wherein those
in control of the social media accounts become the default voice of, and
ideological leaders, of the group. Gerbaudo describes 

\begin{quote}
``the emergence of new forms of power stratification embedded in the hierarchy
of content management systems used by activists, and the explosion of power
struggles for the control of social media accounts''~\cite{gerbaudo2017social}.
\end{quote}

However, the issue does not stop at secrets or accounts. Indeed, these cases are
specific instances of a more general problem: hierarchical access control. In
current access control systems, whether or not an entity should be allowed to
access a resource is decided by a hierarchical process. In discretionary access
control the owner of an object decides whether or not others can read from,
write to, or execute that object. With the recognition that some files are more
important to operations than others, it is easy to see that this creates a
hierarchy. This access control information itself must be stored in an object,
and that object must have an owner. Thus, the owner of that object has the
ability to change the system and reallocate permissions, putting themselves on
top of that hierarchy. Similarly, in Role Based Access Control at least one role
needs to be able to assign and remove roles, allowing them to grant themselves
whatever role they need to access to any object they wish. More generally, the
authorization process of access control is centralized at the top of the
hierarchies of each individual organization.

In his work \textit{The Moral Character of Cryptographic Work}, Rogaway states
``Cryptography can be developed in directions that tend to benefit the weak or
the powerful.''~\cite{rogaway2015moral}. However, there is no reason why this
must stop at cryptography, and cannot generalize to all the fields of security.
Why, then, have we not developed \textbf{more} technologies that benefit the
community by limiting the privileges of the powerful within an organization via
democratic participation? Why have no access control methods been developed to
allow for authorization via horizontal, participatory protocols?

In this work we address this gap. In the following sections we propose COLBAC, a
collective based system that allows for access control via flexible and dynamic
democratic participation to fit the needs of the organization using it. We start
by discussing the current hierarchical state of access control in Section
\ref{sec:definition} and discussing previous attempts at hierarchical security
in Section~\ref{sec:previous}. We then discussion COLBAC's threat model in
Section~\ref{sec:threat_model} and introduce the design goals of COLBAC in
Section~\ref{sec:colbacrequirements}. We then informally describe the COLBAC
system design in Sections~\ref{sec:colbacdesign}, \ref{sec:Tokentypes} and 
\ref{sec:Tokenformat}, and formally define COLBAC in
Section~\ref{sec:colbacformal}. We then discuss the properties of COLBAC in
Section~\ref{sec:colbacproperties} and discuss the intersection of Democracy and
COLBAC in Section~\ref{sec:democracy}. In Section~\ref{sec:security_analysis} we
perform a security analysis of COLBAC and in Section~\ref{sec:usability} we
discuss its usability and scalability. We then discuss COLBAC's limitations
in Section~\ref{sec:limitations}. Finally, we discuss open research and future
work in Section~\ref{sec:future_work} and in Section~\ref{sec:conclusion} we
conclude.
%organizations~\cite{kavada2020counterpublics}.
%
%\begin{quotation}
%\textit{Yet, the material design and regulation of social media platforms soon
%put anonymous administrators into leadership positions. In both movements,
%social media gave the administrators a public stage to distribute messages to
%large numbers of people and exclusive access to the metrics and demographics of
%user engagement. On Facebook pages, for instance, admin posts are displayed on
%the main timeline and are directly visible to users. On large pages... user
%comments are... like a continuous stream with the admin post as the frame. Admin
%posts are also written in a collective voice since administrators speak as [the
%collective]. This provides administrators with significant power as they can
%embody the collective voice of the movement.}
%\end{quotation}
%
%Kavada then discusses the concept of a ``password war,'' or a series of
%interactions in which administrators of groups attempt to lock each other out of
%the group social media page by changing passwords. This may harm the perceived
%legitimacy of the movements, since one may believe that it represents that the
%organization cannot achieve its own ideals. Worse still, these struggles set
%the organization back: they must spent time to rectify a power dynamic that was
%previously resolved through democratic participation. In this respect,
%hierarchical tools introduce inefficiencies into the operation of democratic
%organizations.
%
%This design issue also leads to the ``digital vanguard'' problem outlined by
%Paulo Gerbaudo: in so-called ``leaderless'' social movements, those participants
%who take on the role of managing the communications platforms end up taking on a
%``vanguard'' role insofar as they are able to set the tone and agenda of the
%movement through these communications~\cite{gerbaudo2017social}. Gerbaudo’s,
%Kavada's, and other % will change to our own after reviews. It gives a hint to who is authoring the work.
%ethnographic research confirms that bitter struggles have occurred in many of
%these non-hierarchical social movements and democratic political groups over 
%control of passwords and administration of platforms, sometimes accelerating to
%the point of lawsuits.
%
%However, this may not be the fault of the organizations themselves. How can an 
%organization achieve the goals of horizontality and equality when the tools they
%must use for organization assume hierarchy, and grant some individuals more
%control than others? Must these tools be this way? Are there better ways to
%design tools to account for horizontality and equality in a client organization?
%If a tool does not allow for horizontality and equality in a client
%organization, can another tool sit in between the offending tool and the
%organization to address the issue? 
%
%Indeed, we know these concepts can be placed into practice, as they have before
%with horizontal security technologies such as PGP and its web of trust. However,
%where horizontal tools do exist, they tend to be rigid and inflexible, only
%allowing for a pre-defined notion of horizontality. Examples of this are
%technologies that require a set percentage of the system to agree to reach a
%consensus, instead of allowing the percentage required to be flexible and
%dynamic, adjusting to the needs of the organization at the time. Can these tools
%be modified to allow for dynamic horizontality, responding to the needs of an
%organization at a given moment?
%
%In his work \textit{The Moral Character of
%Cryptographic Work}, Rogaway states ``Cryptography can be developed in
%directions that tend to benefit the weak or the
%powerful.''~\cite{rogaway2015moral}. Why, then, have we not developed
%\textbf{more} technologies that benefit the community by limiting the privileges
%of the powerful within an organization via democratic participation?
%
%In this paper we argue for the construction of horizontal security tools and
%techniques that are flexible and respond to the needs of an organization at a
%given time. We explore different potential approaches to horizontal security,
%and examine some examples of horizontal security practices currently in use. We
%provide an example of the idea of dynamic horizontality with an access control
%model called COLBAC, or collective based access control, which includes dynamic
%and flexible horizontality to fit the needs of the organizations using it. We 
%then address some potential benefits and drawbacks of horizontal security, and
%the effects it may have on privacy. We conclude with future work, and supply a
%set of unsolved problems that must be addressed to realize our vision of
%horizontal security.

The struggle of power between centralized, hierarchical structures and
decentralized, horizontal structures has existed for millennia. This struggle has
expressed itself in all forms of life: politics, economics, social interaction,
and more. Recent events all around the globe have demonstrated the resurgence of
a desire for more inclusive, participatory, and fair social structures. From the
demands of Black Lives Matter protesters regarding police oversight and funding,
to the raise in popularity in worker cooperatives that allow workers to have
power over the product of their labor, people are continuing to demand more
participation in important economic and societal decision-making processes. %TODO: CITE ME
Even some schools are beginning to include the democratic participation of the
school children to their decision making processes, encouraging engagement and
investment in the child's own education.~\cite{pacheco2008escola}

This shift in organization has potential implications in how organizations
interact internally, with each other, and with society as a whole. Whereas the
interactions between hierarchical organizations are overwhelmingly guided by
the visions of the individuals who are on top of their respective organization's
hierarchies, interactions of more horizontal organizations are more complex, and
are meant to reflect the will of the entire group that make up the organization.
In addition, responsibilities in more horizontal organizations tend to shift.
For example, in worker cooperatives individuals often shift between different
job positions in order to learn how the company as a whole functions, and to
keep their work fresh and 
interesting~\cite{jackall19846, wright2014worker, lindenfeld1982workplace}.
To complicate matters further, different organizations have different, and
sometimes dynamic, levels of  horizontality~\cite{wright2014worker}. Some decide
to shift all decision making power to the collective, making approval for all
important actions require democratic votes or consensus. Others select
representatives or working groups to delegate specific tasks, allowing for
quicker actions and responses to events, but within a power structure that
allows for the revocation of those who misuse their power. Some organizations
take a hybrid approach, requiring democratic participation of the whole
organization for some issues, and electing working groups for others. Some
organizations sway between all of the above listed approaches.

Despite the rapid demand of this changed structure of organization, 
Cybersecurity tools, technologies and techniques continue to remain highly 
hierarchical. Security policies are written by a security team or imported from
pre-existing (and sometimes external) templates and imposed on the rest of the
organization, with at most the limited oversight and approval of a manager or
higher-up. Even still, not every member of the security teams are participating
in deciding what these policies are meant to achieve. This decision is often
made by a manager, who ensures that the policies do not get in the way of the
visions of the individuals in the chief officer positions. {\color{red}TODO: CITE ME} These technologies
and policies reflect political decisions made during the creation of these
processes and technologies, and the technologies and policies themselves have
political implications~\cite{winner1980artifacts}.

More, secrets such as organizational passwords to third-party services are often
trusted to one individual or small group in an organization, but easily
recovered by an administrator or other individual higher in the hierarchy in the
case that the individual responsible leaves the organization or performs a
violation. However, when an organization becomes less hierarchical, the decision
of who should have access to such secrets becomes more difficult to make. If
everyone should have access to the secrets, such as passwords, then any insider
can change the passwords and cause denial of service for the organization. If
only a few people have the password, then a hierarchy forms based on one's
access to confidential organizational material. As Kavada states, administrators
hold great power over their respective
organizations~\cite{kavada2020counterpublics}.

\begin{quotation}
\textit{Yet, the material design and regulation of social media platforms soon
put anonymous administrators into leadership positions. In both movements,
social media gave the administrators a public stage to distribute messages to
large numbers of people and exclusive access to the metrics and demographics of
user engagement. On Facebook pages, for instance, admin posts are displayed on
the main timeline and are directly visible to users. On large pages... user
comments are... like a continuous stream with the admin post as the frame. Admin
posts are also written in a collective voice since administrators speak as [the
collective]. This provides administrators with significant power as they can
embody the collective voice of the movement.}
\end{quotation}

Kavada then discusses the concept of a ``password war,'' or a series of
interactions in which administrators of groups attempt to lock each other out of
the group social media page by changing passwords. This may harm the perceived
legitimacy of the movements, since one may believe that it represents that the
organization cannot achieve its own ideals. Worse still, these struggles set
the organization back: they must spent time to rectify a power dynamic that was
previously resolved through democratic participation. In this respect,
hierarchical tools introduce inefficiencies into the operation of democratic
organizations.

This design issue also leads to the ``digital vanguard'' problem outlined by
Paulo Gerbaudo: in so-called ``leaderless'' social movements, those participants
who take on the role of managing the communications platforms end up taking on a
``vanguard'' role insofar as they are able to set the tone and agenda of the
movement through these communications~\cite{gerbaudo2017social}. Gerbaudo’s,
Kavada's, and other % will change to our own after reviews. It gives a hint to who is authoring the work.
ethnographic research confirms that bitter struggles have occurred in many of
these non-hierarchical social movements and democratic political groups over 
control of passwords and administration of platforms, sometimes accelerating to
the point of lawsuits.

However, this may not be the fault of the organizations themselves. How can an 
organization achieve the goals of horizontality and equality when the tools they
must use for organization assume hierarchy, and grant some individuals more
control than others? Must these tools be this way? Are there better ways to
design tools to account for horizontality and equality in a client organization?
If a tool does not allow for horizontality and equality in a client
organization, can another tool sit in between the offending tool and the
organization to address the issue? 

Indeed, we know these concepts can be placed into practice, as they have before
with horizontal security technologies such as PGP and its web of trust. However,
where horizontal tools do exist, they tend to be rigid and inflexible, only
allowing for a pre-defined notion of horizontality. Examples of this are
technologies that require a set percentage of the system to agree to reach a
consensus, instead of allowing the percentage required to be flexible and
dynamic, adjusting to the needs of the organization at the time. Can these tools
be modified to allow for dynamic horizontality, responding to the needs of an
organization at a given moment?

In his work \textit{The Moral Character of
Cryptographic Work}, Rogaway states ``Cryptography can be developed in
directions that tend to benefit the weak or the
powerful.''~\cite{rogaway2015moral}. Why, then, have we not developed
\textbf{more} technologies that benefit the community by limiting the privileges
of the powerful within an organization via democratic participation?

In this paper we argue for the construction of horizontal security tools and
techniques that are flexible and respond to the needs of an organization at a
given time. We explore different potential approaches to horizontal security,
and examine some examples of horizontal security practices currently in use. We
provide an example of the idea of dynamic horizontality with an access control
model called COLBAC, or collective based access control, which includes dynamic
and flexible horizontality to fit the needs of the organizations using it. We 
then address some potential benefits and drawbacks of horizontal security, and
the effects it may have on privacy. We conclude with future work, and supply a
set of unsolved problems that must be addressed to realize our vision of
horizontal security.

Democracy is colloquially defined as ''government by the people.`` The term has
its root in the Ancient Greek demos, which meant both the village (demos, the
smallest possible administrative unit in the state), and ``the people'' (Demos,
which, in ancient Athens, meant anyone who could participate in the governing
Assembly -- adult men, not slaves, who had completed their military training,
and whose parents were from Athens.) In Athenian democracy, the Assembly
practiced direct democracy, wherein the Assembly made decisions through
deliberative processes and without elected representatives. [CITEME] Forms of
governance in which no one has more political or managerial power than any other
are sometimes called horizontal, drawing from the Spanish horizontalidad, which
was first used to describe new political practices in Argentina after a popular
rebellion in 2001~\cite{sitrin2012everyday}. In most contemporary democracies,
most governance at the national level is performed by representative democracy,
wherein the citizens vote for leaders who are entrusted with the work of running
the state. These two forms – representative and direct/horizontal democracy --
can be seen as the most basic types of democracy. Many, many democratic
practices and groups exist which mix these two forms in different ways, and
which use different techniques for decision-making, and the distribution of
power and resources. Furthermore, a plentitude of solutions and debates exist
around the democratic ``boundary problem'' (CITEME), wherein the definition of
who constitutes the Demos is differently decided, and, certainly, has changed
over time.\footnote{For example, almost all self-identifying democratic states
now allow women to vote. The question of citizenship and national borders as
definitive of a Demos is becoming contested in current times of increased
migration, and given the problem of governing objects of a global nature, such
as communication technologies, economies, and climate change.}

While we do not have the space here to delve into a history of democracy and
debates over its various forms, a few key concepts are important to outline.
Decision making within democracies can take many forms. Consensus refers to a
process of decision-making wherein the group comes to a shared agreement on a
decision (everyone agrees (unanimity), or, more commonly, a certain threshold of
agreement is reached (75\% of the group agrees, for example), or, no one
disagrees enough to block a decision in going forward.)\footnote{Consensus is
used in Quaker communities, by the Haudenosaunee (Iroquois) Confederacy Grand
Council, in Indonesia native cultures. It was used in the 2015 United Nations
climate change conference, and the 2020 French Citizen’s Assembly for the
Climate, and has also been used by many political groups and movements such as
the SNCC, the alter-globalization movment of the 1990s, and the Occupy and 15M
momvements.} Deliberation is a process of discussion and argument that is
usually considered essential for consensus and healthy for all modes of
democracy. Deliberation ideally involves authentic, nonhierarchical
conversations, a plurality of participants, and the incorporation of relevant
knowledge and information on the subject under debate, undergirded by the idea
that such discussions allow for people to learn, change their minds, and for a
general wisdom to emerge that is agreeable to the Demos and greater than the sum
of its parts (individual positions). Neither consensus-based democracy nor
deliberative democracy necessarily require voting, but sometimes they employ it.
Voting can be defined as a process by which an individual registers their
preferences (on an issue, for a representative, regarding where to order lunch,
etc.) via some system for counting (a ballot, a stone, a raised hand, an online
form). Election is the selection of political representatives through a voting
process. Sortition is a process for choosing representatives or assembly members
through randomization (lottery). Sortition was used in ancient Athens to choose
the city council and juries, and is currently used by many Citizens’ assemblies
in the EU and UK, and to form juries in many countries. 

In all these cases, technologies have always performed an important role. Tools
for counting, randomization, and deliberation (sortition machines, amplification
architectures) were important to ancient democracies. Technologies and political
practices have changed together; voting machines and mass media have been key to
practicing (and perhaps compromising or warping) representative democracy. More
cutting-edge developments (such as decidim, democracy.os) use online tools and
AI to aid in projects like participatory budgeting, referenda, and direct
decision making and have been adopted by organizations, social movements,
municipal and even national governments (including Barcelona, the French
National Assembly referendum platform, New York City, and many others.) While
platforms exist for sharing power in a variety of forms, very little has been
done to implement horizontality at the security/access control layer.  

To our knowledge COLBAC is the first attempt to incorporate democratic processes
into access control. However, democracy is a complex concept, and COLBAC only
incorporates some aspects of democracy. For example, Sortition is not utilized
at all in COLBAC. For our definition of the Demos, we consider all members of
a system using COLBAC do be part of the Demos and thus eligable to vote. Our
design of COLBAC also assumes there exists an external channel for Deliberation
that is used by all members of the Demos.

Given these assumptions and the design of COLBAC, we can see that COLBAC allows
for a large variety of different democratic governance schemes. More direct
forms of democracy can be done using Action Tokens, and elective representation
can be done using Delegation Tokens. The spectrum between full consensus and
majority decision-making is represented through the secuirty parameter $f$ and
the decision of how much participation is required per vote and, therefore, how
difficult it is to create a quorom can be fine-tuned using the security
parameter $m$.

Though horizontal security may not require us to redefine the three pillars of
Cybersecurity -- confidentiality, integrity, and availability -- it does create
a few security policy and system design implications. Primarily, if meaningful
participation in the creation of security policies, implementation of security
mechanisms, and decisions around enforcement are to occur, more transparency and
participatory procedures are required. If a larger percentage of an organization
or group is participating policy making, implementation, and enforcement 
decisions, all of the information required for them to consider their options
and make their best decision must be available to them. This implies that the
practices of the group or organization must be transparent for all of the
decision-making stakeholders to see, scrutinize, and criticize. This implication
must be reflected in the security policies relating to confidentiality of
relevant information in the organization or group.

A second implication of horizontal security is its aversion to centralized
implementation. As discussed in Section~\ref{sec:definition}, if any of policy
making, implementation, or enforcement is centralized, the horizontality of the
organization's security suffers. This forces us to rethink the idea of the
highly privileged administrator or super user on which most modern operating
systems are build. As discussed previously, such administrative power leads to
centralization and could cause password wars, as discussed by 
Kavada~\cite{kavada2020counterpublics}, or digial vanguards, as discussed by 
Gerbaudo {\color{red} TODO: CITE ME}.

However, this raises issues. Without administrators, what would something like
an operating system look like? Horizontal security implies a change to the 
fundamentals of system building, necessitating the redesign of operating system
permissions. Mainly, if an administrator attempts to perform an action that may
drastically change the security properties of the system, i.e. a change in
security policy, or the installation of a new security mechanism, this change
must be able to be overridden via a democratic process. Said a different way,
actions of administrators must be reversible by a majority of stakeholders, and
the operating system itself must provide a mechanism to support this. This idea
is explored further in Section~\ref{sec:colbac}.

The combination of transparency and stakeholder override implies an interesting
potential opened up by horizontal security: community control. Our proposed
definitions of horizontal security includes a set of stakeholders, but does not
require that those stakeholders exist within the organization. If the users of
an application or service wish to see a detrimental security decision
overturned, horizontal security practices may give them an opportunity to do so.
However, this has some drawbacks. First, it is not immediately obvious what
security techniques, tools, or mechanisms could enforce this. However, this may
be solved with future research and development.

Even with future research and development, however, there are still potential
issues. It may be that users may vote for changes or policy overrides that would
destroy the organization or group. Additionally, it is difficult to perceive any
potential incentive for organizations to allow themselves to be regulated by
users or other community stakeholders. Potential solutions may include limited
user or community stakeholder involvement for certain issues, such as privacy.
This is explored further in Section~\ref{sec:privacy}. If such technology could
be created, and incentives aligned, then such a change could allow for easier
regulation of businesses without the need for understaffed, underfunded, and
incentive-misaligned regulator commissions.

Horizontal security comes with drawbacks as well as opportunities. Given that
more stakeholders take place in the decision making process of a horizontal
organization, more time may be needed for the decision making process. This
may slow the organization, or stop it from taking time-sensitive actions
altogether. This has the potential for opportunity or revenue loss. Thus, an
organization or group may want to consider their flexibility to respond to
quickly-appearing situations when deciding which degree of horizontality works
for them.

One final implication of horizontal security is the need for research and
development. Horizontal security is under-explored, and may need the development
of new tools, techniques, and primitives to realize. More, though some
building-blocks for horizontal security may exist, very few systems or
procedures of this type exist in practice.

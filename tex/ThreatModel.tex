Collective management of resources opens an organization to a collection of both
internal and external threats. As such, it is expected that attackers may be
outsiders impersonating members of a community as well as disgruntled insiders
that are trying to subvert the system. We, however expect that at least a simple
majority of users will act honestly, and will attempt to use the system to 
represent the collective goal of the broader community.

However, we also expect attackers to attempt to inject accounts in attempts to
dilute the democratic process (e.g., a Sybil attack~\cite{sybil1, sybil2}).
While these types of attacks are often performed in the wild~\cite{197120},
we expect different projects to mitigate this threat as they seem fit. One
example of it is the Debian project, for which membership is often gated
through physically-present PGP key signing processes. This has been widely
documented in work by Coleman~\cite{coleman-oss}. 

Malicious users may abuse other elements beyond raw voting power. They could,
for example, gather a group colluding users, who could abuse time-zones (e.g., making decisions and calling votes when many other people are not online), system availability (e.g., network outages,
net-splits) to carry out exceptional votes. This can likely be mitigated by
ensuring the vote parameters are adjusted to the nature of the community (e.g.,
by adjusting simple majority to relative majority or set an arbitrary threshold
for a vote to come through).

One of the largest potential issues that COLBAC faces is lack of usability.
However, before we can discuss methods of improving COLBAC's usability, we first
must analyze it. 

The most obvious usability concern of the COLBAC system is the amount of time
that a given petition phase may take. However, in order to reason about the
usability of COLBAC's petition phase, we first need to define its factors. These
include:
\begin{enumerate}
\item The number of members of a COLBAC based system, $m$.
\item The amount of time COLBAC petition phases are open, $t$.
\item The average amount of time it takes for members to vote, $a$.
\item Fraction of people required to form a quorom, $f$.
\end{enumerate}

%Idea:
%      Assume a few amounts of users and a few values for the quorom. Maybe 10,
%       100, and 1000 members and 25%, 50%, and 75% for quorom?
%      Calculate the average amount of time it takes to vote based on edits to
%       Wikipedia pages.
%      Given the above numbers, calculate amounts of time we expect COLBAC votes
%       to be open.

In COLBAC there is a balance that must be struck between usability and security,
specifically in the amount of time that a COLBAC petition phase is open. We
would like to strike an optimal balance where the COLBAC petition phase is open
long enough for members to vote, but does not cause delay on necessary actions
being performed on the system.

However, the amount of time for the petition phase of a token is not the only
usability or user experience concern of the COLBAC system. Additionally, issues
may arise with voting fatigue. Specifically, large amount of votes on action
tokens could cause users to become fatigued, and thus pay less attention to the
details of each vote. To avoid this, voting can be simplified through the use of
simplified democratic structures like elected representatives. Here we can have
seats for administrating different functions of a system, and these seats can be
elected by democratic vote. This can be done via long-lived delegation tokens
with permissions relevant only to the elect representative's role within the
system. If the representative misbehaves their permissions could be revoked with
an action token, and new elections can be held. Similar to how workers in some
worker cooperatives hire their managers CITEME, or how union members elect their
stewarts CITEME, members in COLBAC can elect their administrators.

For organizations that prefer more direct democratic approaches, issues with
voting fatigue can be avoided using non-technological organizational practices.
For example, many more hierarchically run organizations (such as some of the 
Mondrogon cooperatives in Spain or the democratic free school Escola da Ponte in
Portugal) have regular meetings where organizational decisions are made.
Organizations that run COLBAC can also use these meetings to decide aspects of
system administration. This alleviates voting fatigue, since results of these
discussions can be batched into a few action tokens and the participants know
exactly what they are voting on before the Petition Phase of COLBAC even begins.
%These usability factors can further be simplified through the organizational
%structures that use them. For example, many cooperatives have weekly meetings of
%workers or members to discuss and vote on weekly work. In this circumstance, the
%voting occurs via democratic, offline processes. Then, the vote on COLBAC is
%only a formality; it should reflect the decisions made via the weekly meetings.

The concept of horizontality of security introduces many opportunities for
future work. The first, and possibly most obvious, opportunity is measuring the
horizontality of security in different types of organizations, including typical
corporations, worker cooperatives, trade unions, activist groups, and more.
Once the horizontality of security of these organizations is measured we can
then begin analyzing the security policies, implementation techniques, tools,
and technologies they use to maintain security in their organizations. We can
also compare and contrast policies such as surveillance of workers and clients
to determine whether more horizontality leads to worker privacy gains.

Another opportunity for future work is the development of tools, techniques, and
technologies that facilitate or encourage horizontal security. This can begin by
solving current horizontal security problems, such as the password wars
discussed by Kavada~\cite{kavada2020counterpublics}, and other aspects of 
horizontal secret sharing. One potential solution for this is a tool that acts
as a reference monitor between the members of an organization and that
organization's accounts, and ensures that some sensitive actions, such as
password resets, cannot occur without the explicit authorization of a
configurable fraction of the organization. More, this tool may allow
organizations to vote to undo any detrimental actions taken by system
administrators. This solution may require the use of techniques such as secret
sharing~\cite{shamir1979share}, and technologies such as 
OAuth~\cite{leiba2012oauth}. Our next work in the area of horizontal security
will explore these directions.

However, after this system is finished another problem presents itself: the
server's operating system on which the code is running is built with an 
assumption of hierarchy, with a small set of administrators having access to
stop the program at any time, or  uninstall it, or remove the secrets the 
program is storing so that access to all accounts is lost. Future work will be
needed to explore how we can apply the idea of horizontality down to the
operating system level. This work can study how, if at all, we can create a
horizontally controlled operating system on which software can run.

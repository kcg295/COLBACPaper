A collective based access control system is a step in the direction of
horizontally controlled technology: it provides an access control foundation
that other democratically controlled technologies can build on top of. However,
COLBAC's limitations leads to some interesting questions. Specifically, COLBAC's
inability to represent all methods of democratic and participatory design leads
us to wonder an access control method could be created that covers more modes of
organization and governance. In order to achieve this, it would be necessary to
first understand a few key points. How do horizontally run communities organize
themselves? How do they develop trust? How do they view and interact with
technologies, and how would they interact with more horizontal security
technologies like COLBAC?

Answering these questions requires a mixture of ethnographic research,
theoretical computer security research, systems construction with iterative
design, and user experience and usability research. Thus, the creation of more
horizontal security does not end with this work, but rather this work takes a
small step into a much larger world of horizontal technologies.

The concept of horizontality of security introduces many opportunities for
future work. The first, and possibly most obvious, opportunity is determining
the horizontality of security in different types of organizations, including
typical corporations, worker cooperatives, trade unions, activist groups, and
more. Through ethnographic research we can determine how they organize, develop
trust, and make security decisions in the physical world.

After the ethnographic work begins to generate insights, we can begin to develop
tools, techniques, and technologies that facilitate or encourage horizontal
security. One potential positive outcome of this ethnographic research is
adjusting COLBAC to be flexible enough to fit the democratic practices of
potential COLBAC users. This will allow us to address the some of the well-known
HCI issues of infrastructure~\cite{edwards2010infrastructure} by making
\textit{deep} changes to COLBAC to reflect the users' mental models of
democracy. Through this, we can mitigate the problem of \textit{constrained
possibilities} by adjusting COLBAC's design to incorporate more forms of
democracy. However, this seems to necessitate more configuration options for
COLBAC, potentially requiring more \textit{unmediated interactions}. 

After improving COLBAC based on our findings, we can begin to solve other
current horizontal security problems, such as password wars
\cite{kavada2020counterpublics}, and other aspects of horizontal secret sharing.
However, in addition to the small advancements, one cannot lose sight of the
bigger picture. Larger advances must also be made to allow for a large-scale
shift to horizontal technology. Taking what we learn from our ethnographic work,
we can begin to implement other technology that reflects the organization, trust
development, and security decision making they display in physical matters. We
can also measure how the introduction of horizontal security techniques affect
the organization of the communities, and reflect these changes in new
technologies as well.

More, we must ensure that the solutions we develop are usable. The creation of
a horizontal security system, or any system,  is only useful if people are
willing to use it, and are able to use it effectively. Thus, research into the
user experience and usability of these solutions is a requirement for their
continued growth and development.

Finally, the ultimate aim of our work would be a general system. Can we create a
concrete and parseable language to express different methods of organization
with varying levels of horizontality? If so, can this language cover all
possible forms of governance and organization, or only a subset of them? If this
language were created, could we create a system that takes as input a
description of a governance system in this language, and dynamically adjusts the
access control needs to fit the governance structure? If so, how would this
system operate? What would it output? And can it, itself, be secured from
misuse? Future research examines these questions and more.


One area where horizontal security may have major impact is the are of privacy.
Currently, many security policies have built into it a concept of security
through surveillance. For the supposed benefit of security, many companies
monitor the lives of their workers both inside and outside of the workplace.
Some schools provide laptops laden with spyware to the students they are meant
to be teaching, usually with a justification of protecting the student from the
harm they may find on the Internet. Entire governments, even supposedly
democratic ones, surveil their people and the people of other countries for the
supposed collective interest.

Often, these security policy decisions of whether or not to surveil, and whom
to surveil or not surveil, are made hierarchically. One potential implication
of more horizontal security decision making may be the modification or removal
of surveillance policies in general. This would protect the privacy of the
workers of an organization, the students of a school, or the citizens of a
nation and of the world, who may not wish to be surveilled. Even if surveillance
policies do continue as they are currently, the policies would be informed by
the consent of those who are surveilled, and can be revoked at any point by
democratic vote. This is implication means that individuals will be better able
to live a more private life inside of work, inside of school, and inside of
their civil activity.

However, potential privacy benefits do not end there. In 
Section~\ref{sec:implications} we discussed the concept of control by other
stakeholders, not just the members of a given organization. This possibility
leads to another potential privacy benefit: the regulation of data usage by 
software or services by their users. If a user or a set of users were capable of
directly modifying the security policy of Facebook such that Facebook were not
allow to use user data in certain ways, for example, using liked pages to make
profiling predictions, then the users gain more control over their data, and
thus more privacy.

However, it is not guaranteed that the creation of horizontal security leads to
privacy gains. It may be that stakeholders, for whatever reason, decide to
continue to allow their surveillance for some perceived collective good. One may
argue that this is still a privacy gain, since control of their own data was
achieved. Even in that worst case, stakeholders gain the potential of direct
privacy regulation in the future based on a democratic process.

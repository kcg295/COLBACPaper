In this work we presented the concept of horizontal security and presented
COLBAC, a collective based access control system with dyanmic and flexible
horizontality to fit the needs of the organization. We both informally and
formally described the system, discussed its properties, and outlined its
limitations. We then discussed the potential impact of the introduction of
COLBAC, and concluded with several opportunities for interdisciplinary future
work, including ethnographic research, theoretical computer security research,
system design research, user experience research, and more.


%In this work we motivated the need for horizontal security, defined horizontal
%security, and explored the potential implications of horizontal security, with a
%focus on how such a concept could change the fields of Cybersecurity and 
%privacy. We examined some examples of different horizontal security practices
%and technologies currently in use and discussed the primitives that may be
%needed to build future horizontal security systems. We also introduced COLBAC,
%a collective based  model that allows for democratic participation in access
%control decisions. We then discussed potential future work in the area,
%providing first steps towards solving problems in the area of horizontal
%security.

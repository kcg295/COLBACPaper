Though COLBAC provides a first step into democratic technology and horizontal
security, it still suffers from some setbacks. Like most access control
systems COLBAC does not aim to solve issues regarding account takeover or fake
accounts. Rather, solving these problems is left to outside solutions such as
salted, peppered, and hased passwords and identity confirmation by the
organization running COLBAC. However, if identity verification and account
security are not well handled they could lead to Sybill attacks against the
COLBAC system. This could lead to Tokens that would have failed to move past the
Petition Phase to pass instead, or Tokens that would have passed to fail because
of compromised or fake accounts. 

Additionally, while our work is, to our knowledge, the frist to apply democratic
processes to access control, it does not provide any new defenses to these
democratic processes. If a democratic process is weak to an attack (for example,
a group of individuals refusing to vote to deny the quorom needed to
authenticate a Token, or a Petition Phase being held when the opponents of the
petition are likely to be busy and unable to vote), then COLBAC will also be
vulnerable to that attack. Future work, including ethnographic and technical
work, can explore how to defend against attacks aimed at democratic systems.

More, in this work we do not cover all participatory organizational or
governance schemes. Some governance schemes (such as plein-air citizens
assemblies, or representative democracies that do not permit referendums) cannot
trivially be represented with COLBAC. Future work will explore how to expand the
set of governance schemes COLBAC can represent.

Other limitations of the COLBAC system fall in the area of usability and user
experience. Democratic processes can be difficult for people to learn and
participate in at first CITEME, and while COLBAC does bring democratic processes
to technology, it does not provide a method of making democratic processes any
simpler. Further, there will be a learning curve for individuals moving to the
COLBAC system, even if they are familiar with democratic processes. In order to
effectively use the COLBAC system, one must know what it is they are voting for,
and what the potential outcomes of these votes are. Future work will explore
methods of presenting this information to users such that they can make an
informed decision while voting on a token. Such potential solutions could
include static or dynamic analysis of the code proposed to be run, or performing
"dry runs" of the software on a virtualized environment to determine its
effects.

Ultimately, difficulties introduced by purely horizontal democratic processes
may cause some organizations, specifically larger organizations, to adopt a more
hierarchical elected representative system. However, this is not a limitation of
the scalability of COLBAC, but rather a limitation of scalability of
participatory structures themselves. However, COLBAC addresses this difficulty
by providing the ability for organizations' systems to operate as
representational democracies\footnote{i.e. "elect your admins"}, allowing for
organizations to choose the level of horizontality that works for them at a
given time.

Though COLBAC does have limitations, no system is perfect when it's first
created. Hierarchical access control systems are built on top of a monumental
amount of previous work and lessons learned from previous adoptions. However,
COLBAC is the first attempt at constructing a democratic access control system.
As such, we will build upon the proposed system in future work to solve some of
the security and usability issues it has, and we expect to arrive at a more
usable and secure democratic access control system.

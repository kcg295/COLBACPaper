Social structures around the world are changing to be more inclusive and
democratic. Governance structures are switching away from old dictatorial and
monarchical structures into structures of citizen participation. Some business
structures are beginning to shift from strictly hierarchical organization, such
as that of the traditional corporation or business, to a more democratic
structure of worker ownership. Many activists groups are beginning to run
themselves without hierarchical structure at all, relying on democratic
participation of members to decide which ways they act. Despite this,
Cybersecurity structures lag behind this change, remaining rooted in their
origins in highly hierarchical military structures. In this paper we explore
what it means to design horizontal security. We explore different potential
approaches to horizontal security, and examine some examples of horizontal
security practices currently in use. We then address some potential benefits and
drawbacks of horizontal security, and the effect they may have on an
organization. We then conclude with future work, and how the idea of horizontal
security may be realized in the future.

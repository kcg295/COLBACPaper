Cybersecurity suffers from an oversaturation of centralized, hierarchical
systems and a lack of exploration in the area of horizontal security, or
security techniques and technologies which utilize democratic participation for
security decision-making. Because of this, many horizontally governed
organizations such as activist organizations, worker cooperatives, trade unions, i
and others are not represented in current cybersecurity solutions, and are
forced to adopt hierarchical solutions to cybersecurity problems. This causes
power dynamic mismatches that lead to cybersecurity and organizational
operations problems. In this work we introduce COLBAC, a collective based access
control system aimed at addressing this gap. COLBAC uses democratically
authorized capability tokens to express access control policies. It allows for
a flexible and dynamic level of horizontality to meet the needs of different
horizontally governed organizations. To our knowledge, COLBAC is the first
attempt at a horizontal, democratic, and participatory  access control system.
After introducing COLBAC, we finish with a discussion on future work needed to
realize more horizontal security techniques, tools, and technologies.

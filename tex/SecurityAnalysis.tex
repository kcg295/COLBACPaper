%In this section, we explore the security properties of COLBAC in two possible
%scenarios.  First, Section explores and contrasts how COLBAC's token system
%fares through usual system operations. We expect to develop a taxonomy of
%security-related failure modes for COLBAC enabled systems After, we elaborate
%on possible endpoints related to COLBAC operation that an attacker could abused
%by attackers.

%In this section we explore the security properties of COLBAC. In Section
%\ref{sec:normal_operation} we explore how COLBAC deals with usual system
%operations. Then, in Section~\ref{sec:failure_modes} we develop a taxonomy of
%security-related failure modes for COLBAC enabled systems, then elaborate on
%possible endpoints related to COLBAC operation that an attacker could abuse.

%\subsection{COLBAC failure modes and attack surface}
%\label{sec:failure_modes}
COLBAC's ability to carry out collective decision making is dependent on the
attackers' ability to affect foundational elements of Information Security:
Confidentiality, Availability, Integrity and Non-repudiation. Although COLBAC
is still in the ``theoretical'' space, we are able to explore the failure modes
within the system and derive a taxonomy of potential attacks by malicious users (as per the
categorization outlined in Section~\ref{sec:threat_model}). In this taxonomy,
the potential threats we face are:

\begin{itemize}
    \item {\bf Collective bypass}: a cadre of malicious users is able
        to perform actions bypassing the democratic process.

    \item {\bf Sybil attack}: a malicious user is able
        to impersonate legitimate voters.

    \item {\bf System disruption}: a cadre of malicious users is able to halt
        the system from operating by disrupting voting operations.

\end{itemize}

To explore the attack surface, we enumerate the states and interfaces of
COLBAC described in Sections~\ref{sec:colbacdesign} through 
\ref{sec:colbacformal} and explore the impact if these interfaces were
compromised/mis-implemented.\\

\noindent$DraftToken$\mbox{}\\
Failure in the DraftToken function may materialize by a failure to authenticate
the originator. As a consequence, it may allow malicious users to create tokens
that impersonate actions on behalf of other users. This could lead to a
\emph{Collective bypass} or a \emph{Sybil} attack.\\

\noindent$PetitionToken$\mbox{}\\
Within $DraftToken$, the $Petition$ endpoint may fail as well. In this case, it
is likely that a system stalled waiting for peer input (i.e., votes) may halt
in a similar way as a SYN-flood attack. As such, the $Petition$ function could
be abused to cause a \emph{System Disruption}. Similarly, $Vote$ can be abused
by attackers to flood the system with bogus votes, with a similar consequence.\\

\noindent$AuthorizeToken$\mbox{}\\
Finally, $AuthorizeToken$ likely serves as a gatekeeper for actions running on
the system. An attacker in control of this endpoint is able to bypass any
democratic process (i.e., a \emph{Collective Bypass}). However, in contrast to
failure in $DraftToken$ functions, it is likely that a compromised
$AuthorizeToken$ endpoint could leave no trace of compromise. With this in
mind, we envision this functionality of COLBAC to run as part of a hardened
environment (e.g., a hardware enclave~\cite{sgx}).

In this section, we explore the security properties of COLBAC in two possible
scenarios.  First, Section~\ref{sec:technical} explores some of the possible
technical attacks against the COLBAC system at different system endpoints.
After, we elaborate on some well-known attacks against democracy, and how they
fit into and are mitigated by COLBAC's design.
%fares through usual system operations. We expect to develop a taxonomy of
%security-related failure modes for COLBAC enabled systems After, we elaborate
%on possible endpoints related to COLBAC operation that an attacker could abused
%by attackers.

%In this section we explore the security properties of COLBAC. In Section
%\ref{sec:normal_operation} we explore how COLBAC deals with usual system
%operations. Then, in Section~\ref{sec:failure_modes} we develop a taxonomy of
%security-related failure modes for COLBAC enabled systems, then elaborate on
%possible endpoints related to COLBAC operation that an attacker could abuse.

\subsection{COLBAC failure modes and technical attack surfaces}
\label{sec:technical}
COLBAC's ability to carry out collective decision making is dependent on the
attackers' ability to affect foundational elements of Information Security:
Confidentiality, Availability, Integrity and Non-repudiation. Although COLBAC
is still in the ``theoretical'' space, we are able to explore the failure modes
within the system and derive a taxonomy of potential attacks by malicious users
(as per the categorization outlined in Section~\ref{sec:threat_model}). In this
taxonomy, the potential threats we face are:

\begin{itemize}
    \item {\bf Collective bypass}: a cadre of malicious users is able
        to perform actions bypassing the democratic process.

    \item {\bf Sybil attack}: a malicious user is able
        to impersonate legitimate voters.

    \item {\bf System disruption}: a cadre of malicious users is able to halt
        the system from operating by disrupting voting operations.

\end{itemize}

To explore the attack surface, we enumerate the states and interfaces of
COLBAC described in Sections~\ref{sec:colbacdesign} through 
\ref{sec:colbacformal} and explore the impact if these interfaces were
compromised/mis-implemented.\\

\noindent$DraftToken$\mbox{}\\
Failure in the DraftToken function may materialize by a failure to authenticate
the originator. As a consequence, it may allow malicious users to create tokens
that impersonate actions on behalf of other users. This could lead to a
\emph{Collective bypass} or a \emph{Sybil} attack.\\

\noindent$PetitionToken$\mbox{}\\
Within $DraftToken$, the $Petition$ endpoint may fail as well. In this case, it
is likely that a system stalled waiting for peer input (i.e., votes) may halt
in a similar way as a SYN-flood attack. As such, the $Petition$ function could
be abused to cause a \emph{System Disruption}. Similarly, $Vote$ can be abused
by attackers to flood the system with bogus votes, with a similar consequence.\\

\noindent$AuthorizeToken$\mbox{}\\
Finally, $AuthorizeToken$ likely serves as a gatekeeper for actions running on
the system. An attacker in control of this endpoint is able to bypass any
democratic process (i.e., a \emph{Collective Bypass}). However, in contrast to
failure in $DraftToken$ functions, it is likely that a compromised
$AuthorizeToken$ endpoint could leave no trace of compromise. With this in
mind, we envision this functionality of COLBAC to run as part of a hardened
environment (e.g., a hardware enclave~\cite{sgx}).

\subsection{Democratic Attacks and COLBAC}
%\noindent\textbf{Attacks on Democracy}\mbox{}\\
In addition to the technical attacks enumerated above, there are also political
attacks that can occur against the system. Each of these attacks are not
specific to our system, but rather are attacks against democratic forms of
organization and management.\\

\noindent\textbf{Malicious Representative}\mbox{}\\
One such attack is the concept of a malicious representative, or an entity in
the organization that has gained access to a resource through a Delegation Token
but then does not perform the promised actions, or, worse, performs other
actions that are not in the interest of the organization, causing a
\emph{Collective bypass}. Unfortunately, no technical solution exists to the
problem of malicious representatives. However, previous democracies have
mitigated this problem through accountability in the form of \emph{checks and
balances}.

In order to have checks and balances, a democratic system must have a group that
is responsible for over-ruling the decisions made by potentially malicious
representatives, and that group must have transparency into the actions of the
representatives and the ability to undo the actions these representatives have
done. In COLBAC, the group responsible for over-ruling the decisions made by the
malicious representatives are the users themselves. Transparency is achieved
through the existence of COLBAC's Immutable Sphere, where the elected
representative cannot remove traces of their actions, nor can they deny members
of the system the right to see the Immutable Sphere. If the users have decided
that the representative has acted maliciously, they can call a Petition Phase
for an Action Token that will revoke the representative's rights. They can also
undo any actions that the representative has made through additional Action
Tokens, or elect a new representative to do that via a new Delegation Token.\\

\noindent\textbf{Misuse of Emergency Tokens}\mbox{}\\
Similar to the Malicious Representative attack, individuals in the organization
can use Emergency Tokens to perform an action that is against the collective
will, causing a \emph{Collective bypass} attack. Similar to the Malicious
Representative attack, these attacks are mitigated through checks and balances
within the system.\\

\noindent\textbf{Artificial Quorum through Collusion}\mbox{}\\
An Artificial Quorum attack occurs when a minority group in the organization
attempts to manipulate who is present at a vote to increase the chances of an
unpopular outcome. This can result in a \emph{Collective bypass}. This attack is
likely to occur for organizations that have a low $f$ value, and thus it can
be mitigated through selecting a large enough value $f$ to ensure a large group
is needed for quorum.\\ 

\noindent\textbf{Quorum Denial through Collusion}\mbox{}\\
In direct contrast to an Artificial Quorum attack, a Quorum Denial attack occurs
when a minority group in the organization refuses to log into the system to vote
on a popular Token, potentially denying quorum and resulting in a
\emph{Collective bypass}. This attack is likely to occur for organization that
have a high $f$ value, and thus it can be mitigated through selecting a large
enough value $f$ to ensure a small group cannot deny quorum.
    
% One interesting thing to see here is that two different attacks are enabled by
%   the value of f. One can occur is f is small, and one can occur if f is big.


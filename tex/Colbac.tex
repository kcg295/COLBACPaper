\subsection{COLBAC: Collective Based Access Control}
\label{sec:colbac}
As discussed in previous sections, the choice of certain software, protocols, or
techniques such as X509 certificates or PGP web of trust has implications in the
level of horizontality possible for systems built on top of those software,
protocols, or techniques. How, then, can we build a foundation such that
organizations of different horizontality can use the same foundation and arrive
at much differently structured organizations?

In this section we describe our Collective Based Access Control, or COLBAC: our
approach to a dynamic horizontality access control system. We begin by
describing its requirements, focusing on the dynamic horizontality that
separates COLBAC from other access control systems. We then define the system
itself. Finally, we discuss the properties of this system and the limitations of
this system.

\subsubsection{System Requirements}
\mbox{}\\
COLBAC is aimed at addressing a novel requirement in cybersecurity research:
access control and authorization given an organization with dynamic levels of
horizontal control. Though previous approaches exist for hierarchical access
control (such as MAC, RBAC, etc.), to our knowledge no access control model
exists for organizations of dynamic and flexibile horizontality. To realize
this, our solution must be able to be flexible in terms of horizontality. Said
another way, our system must not assume or define a pre-determined threshold for
horizontal control, i.e., it must not assume majority, or super-majority, or
full consensus is the preferred method of democratic participation. Instead,
the threshold must be configured by the collective itself, and must be able to
be changed when necessary. This allows for rapid temporary centralization of the
system to respond to crises, or to perform a task that requires expertise that
few members of the collective contain. However, these moments of centralization
must be quick to expire and easy to override in order to prevent abuse of
centralized power. Said another way, it must always be easy for the collective
to return to more horizontal control.

However, it makes no sense to have horizontal or democratic control without
transparency. An individual cannot meaningfully vote or otherwise decide on a
practice, or place confidence in a representative, without understanding what
the action is going to take place, and what actions have been taken in the past.
More, if authority is abused, the collective must be able to notice the abuse,
and remove the powers that allowed the abuse to take place. Thus, we can see
that any horizontal access control system requires transparency. Towards this
end, our system must have a method of logging information about the actions of
individuals and the collective that is immutable and available.

\subsubsection{System Design}
\mbox{}\\
COLBAC presents a solution to access control that relies around the collective.
However, as will be discussed later in this section, not all objects on the
system will need to be collectively controlled or administered. As such, we
define three distinct \textit{spheres} of the system, or areas that require
different approaches to access control. These spheres are the \textbf{Collective
Sphere}, the \textbf{User Spheres}, and the \textbf{Immutable Sphere}.

In order to achieve different degrees of horizontality, there must be a portion
of the system that is controlled not by any individual user of the system, but
by some democratic process of the users of the system. We call this portion of
the system the \textbf{Collective Sphere}, as it contains programs, files, and
other resources only accessed based on collective authorization. In any
horizontal system, the administrative functions of the operating system would
need to exist within this portion of the system to allow for true collective
control.

However, not everything should be directly managed by the collective. Individual
users may have their own files and programs they intend to use only in a way
that it does not effect other users of the system or the resources of the
collective\footnote{We can think of these as the home directories of users in
modern Unix-like systems.}. These speheres, called the \textbf{User Speheres},
can use traditional DAC systems like modern Unix-like systems without effecting
the horizontality of the system as a whole.

Finally, to have meaningful control of the system we must have transparency. To
achieve this, a system must have an \textbf{Immutable Sphere}, or a portion of
the file system and programs that cannot be altered once written to, including
by democratic control. This allows for the system to provide append only logs
that are vital to maintaining collective control, as described later in this
section.

\begin{enumerate}
%\item Different Spheres
%    \begin{itemize}
%    \item user sphere
%    \item collective sphere (sudo)
%    \item immutable sphere (cannot be altered, can be read)
%    \end{itemize}
\item Registration
    \begin{itemize}
    \item at least 3 users
    \item each user has an offline key.
    \item specify beginning fraction required for collective action
    \end{itemize}
\item Token: Draft Phase
    \begin{itemize}
    \item User drafts a proposal. Two types, action token and delegation token.
    \item See subsection for proposal design.
    \item user signs the proposal with their key.
    \end{itemize}
\item Token: Petition Phase
    \begin{itemize}
    \item Pass along token to reference monitor, reference monitor passes to
    other users.
    \item If a user agrees, signs and returns to reference monitor.
    \item If a user disagrees, signs a disagreement and returns to reference
    monitor.
    \item User can choose not to vote. Signs a blank vote and returns to
    reference monitor.
    \end{itemize}
\item Token: Action Phase
    \begin{itemize}
    \item When petition phase is over, the number of signatures is compared to
    the fraction.
    \item If >=, perform action, log.
    \item Else, just log.
    \end{itemize}
\end{enumerate}

\subsubsection{Types of Tokens}
\mbox{}\\
\begin{enumerate}
\item Action Token
\item Delegation Token (Working Group Token)
\item Emergency Token
\end{enumerate}

\subsubsection{Token Format}
\mbox{}\\
Header:
\begin{itemize}
\item nonce/Id
\item token type
\item party(s) being authorized
\item petitioning party
\item token expiration
\end{itemize}
Action or Emergency Token Body:
\begin{itemize}
\item Code to be run
\item Permissions needed
\item Comments (optional)
\end{itemize}
Delegation Token Body:
\begin{itemize}
\item Permissions needed
\item Restrictions proposed
\item Comments (optional)
\end{itemize}
Footer:
\begin{itemize}
\item number of signatures
\item signature list
\end{itemize}

\subsubsection{Properties of COLBAC}
\mbox{}\\
\begin{enumerate}
\item flexible and dynamic horizontality through configurable fraction.
\item full transparency of actions taken.
\end{enumerate}

\subsubsection{Limitations of COLBAC}
\mbox{}\\
\begin{enumerate}
\item USABILITY!!!!!!!
\end{enumerate}
